\documentclass{article}

\usepackage[french]{babel}
\usepackage[utf8]{inputenc}

%%%%%%%%%%%%%%%% Lengths %%%%%%%%%%%%%%%%
\setlength{\textwidth}{15.5cm}
\setlength{\evensidemargin}{0.5cm}
\setlength{\oddsidemargin}{0.5cm}

%%%%%%%%%%%%%%%% Variables %%%%%%%%%%%%%%%%
\def\projet{0}
\def\titre{Optimisation thermo-nucléaire contrôlée de la cuisson de la
soupe aux choux}
\def\groupe{0}
\def\equipe{0}
\def\responsible{aeinstei}
\def\secretary{mcurie}
\def\others{hlorentz, mplanck, nbohr}

\begin{document}

%%%%%%%%%%%%%%%% Header %%%%%%%%%%%%%%%%
\noindent\begin{minipage}{0.98\textwidth}
  \vskip 0mm
  \noindent
  { \begin{tabular}{p{7.5cm}}
      {\bfseries \sffamily
        Projet \projet} \\ 
      {\itshape \titre}
    \end{tabular}}
  \hfill 
  \fbox{\begin{tabular}{l}
      {~\hfill \bfseries \sffamily Groupe \groupe\ - Equipe \equipe
        \hfill~} \\[2mm] 
      Responsable : \responsible \\
      Secrétaire : \secretary \\
      Codeurs : \others
    \end{tabular}}
  \vskip 4mm ~

  ~~~\parbox{0.95\textwidth}{\small \textit{Résumé~:} \sffamily Ce
  projet consiste à mettre en place un mécanisme basé sur un
  accélérateur de particules, afin de générer un trou noir permettant
  de cuire une soupe aux choux par rayonnement Hawking. La première
  partie décrit les détails de construction de l'accélérateur, tandis
  que la seconde partie s'intéresse aux suites possibles de
  l'ingestion de la soupe. }
  \vskip 1mm ~
\end{minipage}

%%%%%%%%%%%%%%%% Main part %%%%%%%%%%%%%%%%
\section*{Recette}

\end{document}
