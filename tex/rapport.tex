\documentclass{article}

\usepackage[french]{babel}
\usepackage[utf8]{inputenc}

%%%%%%%%%%%%%%%% Lengths %%%%%%%%%%%%%%%%
\setlength{\textwidth}{15.5cm}
\setlength{\evensidemargin}{0.5cm}
\setlength{\oddsidemargin}{0.5cm}

%%%%%%%%%%%%%%%% Variables %%%%%%%%%%%%%%%%
\def\projet{1}
\def\titre{M\'ethodes de calcul numérique/Limites de la machine}
\def\groupe{4}
\def\equipe{18224}
\def\responsible{gardouin}
\def\secretary{mhajjilaamo}
\def\others{electakone, hbastien, acattarin}

\begin{document}

%%%%%%%%%%%%%%%% Header %%%%%%%%%%%%%%%%
\noindent\begin{minipage}{0.98\textwidth}
  \vskip 0mm
  \noindent
  { \begin{tabular}{p{7.5cm}}
      {\bfseries \sffamily
        Projet \projet} \\ 
      {\itshape \titre}
    \end{tabular}}
  \hfill 
  \fbox{\begin{tabular}{l}
      {~\hfill \bfseries \sffamily Groupe \groupe\ - Equipe \equipe
        \hfill~} \\[2mm] 
      Responsable : \responsible \\
      Secrétaire : \secretary \\
      Codeurs : \others
    \end{tabular}}
  \vskip 4mm ~


 %%%%%%%%%%%%%%%%%%%%%%%%%%%%%%%%%%%%%%%%%%%%%%%%%%%%%%%%%%%%%%%%%%% Présentation du travail
  ~~~\parbox{0.95\textwidth}{\small \textit{Résumé~:} \sffamily Ce
  projet consiste à mettre en place un mécanisme basé sur un
  accélérateur de particules, afin de générer un trou noir permettant
  de cuire une soupe aux choux par rayonnement Hawking. La première
  partie décrit les détails de construction de l'accélérateur, tandis
  que la seconde partie s'intéresse aux suites possibles de
  l'ingestion de la soupe. }
  \vskip 1mm ~

  
  %%%%%%%%%%%%%%%%%%%%%%%%%%%% motivation du projet


  %%%%%%%%%%%%%%%%%%%%%%%%%%%% lien avec le cours


  %%%%%%%%%%%%%%%%%%%%%%%%%%%% part du projet réalisée

  
%%%%%%%%%%%%%%%% Main part %%%%%%%%%%%%%%%%
\section*{Recette}


%%%%%%%%%%%%%%%%%%%%%%%%%%%%%%%%%%%%%%%%%%%%%%%%%%%%%%%%%%%%%%%%%%%%%%%% Code

%%%%%%%%%%%%%%%%%%%%%%%%%%%%%%%%%%%%%%%%%%%%%%%%%%%%%%%%%%%%%%%% chapeau d'introduction de la partie 1

  %%%%%%%%%%%%%%%%%%%%%%%%%%%%%%%%%%%%%%%%%%%%%%%%%%%%%%%%%%%%%%% Partie 1


  %%%%%%%%%%%%%%%%%%%%%%%%%%%%%%%%% repésentation décimale réduite
 
  %%%%%%%%%%%%%%%%%%%%%% description

  %%%%%%%%% nom de la fonction

  %%%%%%%%% définition de chacune des variables

  %%%%%%%%% exemples démontrant le bon fonctionnement du code (repésentation graphique si possible)
  
  %%%%%%%%%%%%%%% Si fonction principale

  %%%%%%%%%% Estimation de la complexité

  %%%%%%%%%% évaluation du comportement dans les cas limites



  %%%%%%%%%%%%%%%%%%%%%%%%%%%%%%%%% addition et multiplication en représation décimale réduite
 
  %%%%%%%%%%%%%%%%%%%%%% description

  %%%%%%%%% nom de la fonction

  %%%%%%%%% définition de chacune des variables

  %%%%%%%%% exemples démontrant le bon fonctionnement du code (repésentation graphique si possible)
  
  %%%%%%%%%%%%%%% Si fonction principale

  %%%%%%%%%% Estimation de la complexité

  %%%%%%%%%% évaluation du comportement dans les cas limites


  
  %%%%%%%%%%%%%%%%%%%%%%%%%%%%%%%%% erreur relative réalisée en ajoutant y à x
 
  %%%%%%%%%%%%%%%%%%%%%% description

  %%%%%%%%% nom de la fonction

  %%%%%%%%% définition de chacune des variables

  %%%%%%%%% exemples démontrant le bon fonctionnement du code (repésentation graphique si possible)
  
  %%%%%%%%%%%%%%% Si fonction principale

  %%%%%%%%%% Estimation de la complexité

  %%%%%%%%%% évaluation du comportement dans les cas limites


  
  %%%%%%%%%%%%%%%%%%%%%%%%%%%%%%%%% erreur relative réalisée en multipliant x et y
 
  %%%%%%%%%%%%%%%%%%%%%% description

  %%%%%%%%% nom de la fonction

  %%%%%%%%% définition de chacune des variables

  %%%%%%%%% exemples démontrant le bon fonctionnement du code (repésentation graphique si possible)
  
  %%%%%%%%%%%%%%% Si fonction principale

  %%%%%%%%%% Estimation de la complexité

  %%%%%%%%%% évaluation du comportement dans les cas limites


  
  %%%%%%%%%%%%%%%%%%%%%%%%%%%%%%%%% graphe des fonctions précédentes
 
  %%%%%%%%%%%%%%%%%%%%%% description

  %%%%%%%%% nom de la fonction

  %%%%%%%%% définition de chacune des variables

  %%%%%%%%% exemples démontrant le bon fonctionnement du code (repésentation graphique si possible)
  
  %%%%%%%%%%%%%%% Si fonction principale

  %%%%%%%%%% Estimation de la complexité

  %%%%%%%%%% évaluation du comportement dans les cas limites


%%%%%%%%%%%%%%%%%%%%%%%%%%%%%%%%%%%%%%%%%%%%%%%%%%%%%%%%%%%%%%%% chapeau d'introduction de la partie 2
  

  %%%%%%%%%%%%%%%%%%%%%%%%%%%%%%%%%%%%%%%%%%%%%%%%%%%%%%%%%%%%%%% Partie 2


  %%%%%%%%%%%%%%%%%%%%%%%%%%%%%%%%% Logarithme népérien de 2
 
  %%%%%%%%%%%%%%%%%%%%%% description

  %%%%%%%%% nom de la fonction

  %%%%%%%%% définition de chacune des variables

  %%%%%%%%% exemples démontrant le bon fonctionnement du code (repésentation graphique si possible)
  
  %%%%%%%%%%%%%%% Si fonction principale

  %%%%%%%%%% Estimation de la complexité

  %%%%%%%%%% évaluation du comportement dans les cas limites



  %%%%%%%%%%%%%%%%%%%%%%%%%%%%%%%%% algorithmes CORDIC
 
  %%%%%%%%%%%%%%%%%%%%%% Calculatrice

  %%%%%%%%% représentation des nombres utilisés sur une calculatrice

  %%%%%%%%% avantages de ce genre de représentation

  %%%%%%%%% inconvénients de ce genre de présentation

  
 
  %%%%%%%%%%%%%%%%%%%%%%% algorithmes pour calculer les fonctions exp et trigonométriques

  %%%%%%%%% technique générale utilisée

  %%%%%%%%% efficacité lorsqu'elle est ramenée à une calculatrice


  
  %%%%%%%%%%%%%%%%% implémentation des quatres algorithmes

  
  %%%%%%%%%%%% exp et ln
 
  %%%%%%%%%%%%%%%% description

  %%%%%%%%% nom de la fonction

  %%%%%%%%% définition de chacune des variables

  %%%%%%%%% exemples démontrant le bon fonctionnement du code (repésentation graphique si possible)
  
  %%%%%%%%%%%%%%% Si fonction principale

  %%%%%%%%%% Estimation de la complexité

  %%%%%%%%%% évaluation du comportement dans les cas limites


  
  %%%%%%%%%%%%%%%% tan et arctan
 
  %%%%%%%%%%%%%%%%%%%%%% description

  %%%%%%%%% nom de la fonction

  %%%%%%%%% définition de chacune des variables

  %%%%%%%%% exemples démontrant le bon fonctionnement du code (repésentation graphique si possible)
  
  %%%%%%%%%%%%%%% Si fonction principale

  %%%%%%%%%% Estimation de la complexité

  %%%%%%%%%% évaluation du comportement dans les cas limites



  %%%%%%%%%%%%%%%%%%%%% vérification des algorithmes par la méthode ?


%%%%%%%%%%%%%%%%%%%%%%%%%%%%%%%%%%%%%%%%%%%%%%%%%%%%%%%%%%%%%%%%%% conclusion

  %%%%%%%%%%%%%%%%%%%%%%%%%%% commentaire <participant>



  %%%%%%%%%%%%%%%%%%%%%%%%%%% commentaire <participant>


  
  %%%%%%%%%%%%%%%%%%%%%%%%%%% commentaire <participant>



  %%%%%%%%%%%%%%%%%%%%%%%%%%% commentaire <participant>



  %%%%%%%%%%%%%%%%%%%%%%%%%%% commentaire <participant>

  
  
\end{minipage}

\end{document}
